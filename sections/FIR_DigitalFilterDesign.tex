\section{FIR Digital Filter Design\buch{Chapter 10}}

\subsection{Windowing Methods}
\subsubsection{Ideal Filters}
\subsubsection{Rectangular Window}
\subsubsection{Hamming Window}

\subsection{Kaiser Window}
\begin{align*}
	\alpha = \left\{
		\begin{array}{l l}
			0.1102(A-8.7)& \quad \text{if } A \geq 50\\
			0.5842(A-21)^{0.4}+0.07886(A-21) & \quad \text{if }21 < A < 50\\
			0 & \quad \text{if } A \leq 21
		\end{array} \right.
\end{align*}
\subsubsection{Kaiser Window for Filter Design}


\textbf{Regeln:}\\
pass und stopp Band sind identisch, sind sie unterschiedlich vorgegeben muss die härtere Bedingung verwendet werden \\
\begin{align*}
\delta=min(\delta_{pass},\delta_{stop})
\end{align*}
schlussendlich wird der Filter in beiden Bändern die gleiche Überschreitung haben
\begin{align*}
A=-20log_{¶10}\delta && \delta=10^{-A/20}
\end{align*}
Das Stopp band hat meistens die strengeren Anforderungen
\subsubsection{Kaiser Window for Spectral Analysis}
\subsection{Frequency Sampling Method}

\subsection{Other FIR Design Methods}

