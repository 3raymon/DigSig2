\section{Interpolation, Decimation and Oversampling\buch{Chapter 12}}
\subsection{Interpolation and Oversampling\buchSeite{532ff}}
Oversampling is increasing the sample rate and requires some form of interpolation. New samples at a higher sampling rate are created. These are calculated using a FIR filter. The spectrum of the low-rate and the high-rate signal are identical, if the frequency axis is not normalized.

\subsection{Interpolation Filter Design\buchSeite{638ff}}
\subsubsection{Direct form}
%TODO Fig. 12.2.1
There are now two discrete times, a fast one $n'$ and a slow one called $n$.\\
The ideal L-fold interpolation filter is a lowpass filter operating at the fast sampling rate $f'_s=Lf_s$ and having its cutoff frequency at the low-sampling rates Nyquist frequency $f_s/2$.\\
$f_c=\frac{f_s}{2}=\frac{f'_s}{2L}$\\
$\omega'_c=\frac{2\pi f_c}{f'_s}=\frac{\pi}{L}$\\
FIR approximation to the ideal interpolator (non causal): $d(k')=\frac{\sin(\pi k'/L)}{\pi k'/L}$, with $-LM\leq k'\leq LM$\\
By delaying this filter by LM samples, it gets causal: $d(k')=\frac{\sin(\pi (n'-LM)/L)}{\pi (n'-LM)/L}$\\
Using a Hamming window: $h(n')=w(n')d(n'-LM)$, with $w(n') = 0.54-0.46\cos(\frac{2\pi n'}{N-1})$\\
The output of the non causal FIR interpolator filter is obtained as the convolution of the upsampled signal with the filter $y_{up}(n') = \sum\limits_{k'=-LM}^{LM}d(k')x_{up}(n'-k')$\\
\subsubsection{Polyphase Form\buchSeite{640ff}}
Better because less multiplications are needed. But it is bullshit to sommarize, therefore see chapter 12, pages 18 to 28.
Nevertheless a little help:\\
The great interpolator $d(k')$ can be splitted up to the polyphase subfilter: $d_i(k) = d(kL+i)$ for $-M\leq k \leq M-1$\\
So the output can be described as $y_{up}(nL+i)=\sum\limits_{k=-M}^{M-1} d_i(k) x_{up}(nL-kL)$\\
Now the output $y_{up}(nL+i)$ can be splittet up to $y_i(n)=y_{up}(nL+i)$\\\\
The nice feature is that this output can be expressed with the original input $x(n)$ so that is:\\ $y_i(n)=\sum\limits_{k=-M}^{M-1} d_i(k) x(n-k)$\\\\
The computational rate in terms of the total number of multiplications per second is:\\
 $R=N (L f_s)=N L f_s \quad \text{(direct form)}$\\
 $R=L (2M f_s)=N f_s \quad \text{(polyphase form)}$\\
\subsubsection{Frequency domain characteristics}
%TODO Figure von Seite 34
A filter $D(f)$ is needed, which removes all the spectral copies inside the high-rate Nyquist band, which were not in the low-rate Nyquist band. Ideal: $X'(f) = D(f)X(f)$.
$D(f)=\begin{cases} L,\ if\ |f|\leq\frac{f_s}{2}\\
0,\ if\ \frac{f_s}{2}\leq |f|\leq\frac{f'_s}{2}
\end{cases}$

\subsection{Linear and Hold Interpolators\buchSeite{657ff}}
\subsubsection{Hold Interpolator}
$d(k')=\begin{cases}
1\text{, if }0\leq k' \leq L -1\\
0\text{, otherwise}\end{cases}$\\
$y_{up}(nL+i)=x(n)$\\
Frequency response: $D(f)=\frac{\sin(\pi f/f_s)}{\sin(\pi f/Lf_s}e^{-j\pi(L-1)f/Lf_s}$
\subsubsection{Linear Interpolator}
$d(k')=\begin{cases}
1-\frac{|k'|}{L}\text{, if }|k'|\leq L -1\\
0\text{, otherwise}\end{cases}$\\
$y_{up}(nL+i)=(1-\frac{i}{L}x(n)+\frac{i}{L}x(n+1)$\\
Frequency response: $D(f)=\frac{1}{L}\left|\frac{\sin(\pi f/f_s)}{\sin(\pi f/Lf_s}\right|^2$
\subsection{Design Examples\buchSeite{661ff}}
\subsection{Decimation and Oversampling\buchSeite{686ff}}
\subsection{Sampling Rate Converter\buchSeite{691ff}}
\subsection{Noise Shaping Quantizer\buchSeite{598ff}}
