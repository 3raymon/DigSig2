\section{DFT/FFT Algorithms\buch{Chapter 9}}
\subsection{Begriffe}

\begin{tabularx}{\textwidth}{l p{3cm}XX}
	Abkürzung & Name  & Eigenschaft & Anwendung \\\hline
	DTFT &
	discrete-time Fourier transform &
	bildet ein endliches, zeitdiskretes Signal auf ein kontinuierliches, periodisches Frequenzspektrum ab &
	Im Spektrum lässt sich unter Umständen  abschnittsweise ein geschlossener mathematischer Ausdruck angeben \\
	DFT &
	discrete Fourier transform &
	bildet ein zeitdiskretes, endliches Signal, welches periodisch fortgesetzt wird, auf ein diskretes, periodisches Frequenzspektrum ab &
	DFT besitzt zur Signalanalyse große Bedeutung \\
	FFT &
	fast Fourier transform &
	ein Algorithmus zur effizienten Berechnung der Werte einer diskreten Fourier-Transformation (DFT) &
	\vspace{-19pt}
	\begin{itemize}
		\item Zur Reduzierung des Berechnungsaufwandes bei der zirkularen Faltung
		\item im Zeitbereich von FIR-Filtern
		\item Ersatz durch die schnelle Fouriertransformation und einfache Multiplikationen im Frequenzbereich
		\item Synthese von Audiosignalen aus einzelnen Frequenzen über die inverse FFT
	\end{itemize}\\
\end{tabularx}

\subsection{Frequency Resolution and Windowing\buchSeite{464-472}}
\begin{tabular}{|l|l|l|}
	\hline
	symbol & property & formulas
	\\ \hline
	$T_L$ & duration of data record & $ = LT$
	\\ \hline
	$w(n)$ & Window function &
	\\ \hline
	$\Delta \omega_w$ & Window width (rad/sample) & $ = c \frac{2 \pi}{L}$
	\\ \hline
	$\Delta f_w$ & Window width (Hz) & $ = c \frac{f_s}{L} = c \frac{1}{T_L} $
	\\ \hline
	$\Delta f$ & Frequency resolution & $ \Delta f \geq \Delta f_w=c\frac{f_s}{L}=c\frac{1}{T_L}$
	\\ \hline
	$\Delta\omega$ & Resolvability condition & $=\Delta\omega \geq \Delta\omega_w = c\frac{2\pi}{L}$ 
	\\\hline
	$R$ & relative sidelobe level &
	\\ \hline
	$c$ & depends on window, always $\geq 1$ &
	\\ \hline
\end{tabular}


\begin{flalign*}
& \text{Window/Funktion} && \text{Rectangular} && \text{Hamming}\\
& w(n) &&  \begin{cases}
			1, & \text{if } 0 \leq n \leq L-1 \\
			0, & \text{otherwise}
 \end{cases} &&
\begin{cases}
	0.54 - 0.46 \cos(\frac{2\pi n}{L-1}), & \text{if } 0 \leq n \leq L-1 \\
	0, & \text{otherwise}
\end{cases} \\
& \text{window tradeoff } c && c=1 && c=2\\
& R && -13.46 dB && -40 dB\\
& x_{_L}(n) && x(n)w(n) && \\
& X_{_L}(\omega) && \int_{-\pi}^{\pi}X(\omega')W(\omega-\omega')\frac{d\omega'}{2\pi} && \\
\end{flalign*}



\subsection{DTFT Computation}
\subsubsection{DTFT at a Single Frequency\buchSeite{475-477}}
\begin{align*}
X(\omega)=\sum_{n=0}^{L-1}x(n)e^{-j\omega n}=\sum_{n=0}^{L-1}x(n)z^{-n}\biggr|_{z=e^{j\omega}}=X(z)\biggr|_{z=e^{j\omega}}
\end{align*}

For example let L
\begin{flalign*}
& X && = && x_3+z^{-1}X && = && x_3\\
& X && = && x_2+z^{-1}X && = && x_2+z^{-1}x^3\\
& X && = && x_1+z^{-1}X && = && x_1+z^{-1}x_2+z^{-2}x_3\\
& X && = &&  x_0+z^{-1}X && = && x_0+z^{-1}x_1+z^{-2}x_2+z^{-3}x_3 && = && X(z)&&\\
\end{flalign*}

\subsubsection{DTFT over a Frequency Range\buchSeite{478}}
\begin{align*}
\omega_k=\frac{2\pi k}{N}&& f_k=\frac{kf_s}{N} && z_k=e^{j\omega_k}=e^{2\pi jk/N} && k=0,1,\ldots , N-1 
\end{align*}
\begin{align*}
X(\omega_k)=\sum_{n=0}^{L-1}x(n)e^{-j\omega_kn} = \sum_{n=0}^{L-1}x(n)z_k^{-n} && \text{mit} && z_k=e^{j\omega_k} = z_k=e^{n\pi j k /N}
\end{align*}
\begin{align*}
\omega_k=\omega_a+k\frac{\omega_b-\omega_a}{N}=\omega_a+k\Delta\omega_{bin} && \omega_{bin}=\frac{\omega_b-\omega_a}{N} && \text{or in Hz} && \Delta f_{bin}=\frac{f_b-f_a}{N} && \omega_{k+N}=\omega_k+2\pi
\end{align*}

\subsubsection{DFT\buchSeite{479-481}}
\begin{align*}
\omega_k=\frac{2\pi k}{N}&& f_k=\frac{kf_s}{N} && z_k=e^{j\omega_k}=e^{2\pi jk/N} && k=0,1,\ldots , N-1 
\end{align*}
\begin{align*}
X(\omega_k)=\sum_{n=0}^{L-1}x(n)e^{-j\omega_kn} = \sum_{n=0}^{L-1}x(n)z_k^{-n} && \text{mit} && z_k=e^{j\omega_k} = z_k=e^{n\pi j k /N}
\end{align*}
\begin{align*}
\Delta_{bin}=\frac{2\pi}{N} && \text{or} && \Delta f_{bin}=\frac{f_s}{N} && \omega_{k+N}=\omega_k+2\pi
\end{align*}
\subsubsection{Zero Padding\buchSeite{482}}
Alle an das Signal angehängten '0' ändern die DTFT nicht 
\begin{align*}
x= [x_0,x_1,\ldots,x_{L-1}] && x_D=[x_0,x_1,\ldots,x_{L-1},\underbrace{0,0,\ldots,0}_{D zeros}] && X_D(\omega_k)=X(\omega_k)\\
\end{align*}

Alle vor das Signal angehängten '0' verschieben das Original Signal zurück
\begin{align*}
x= [x_0,x_1,\ldots,x_{L-1}] && x_D=[\underbrace{0,0,\ldots,0}_{D zeros},x_0,x_1,\ldots,x_{L-1}] \\
X_D(\omega)=e^{-j\omega D}X(\omega) && X_D(\omega_k)=e^{-j\omega_k D}X(\omega_k) && k=0,1,\ldots,N-1
\end{align*}

\subsection{Physical versus Computional Resolution\buchSeite{482-485}}

\subsection{Matrix Form of DFT}

\subsection{Module-N reduction}
\begin{itemize}
	\item Opposite of zero padding
\end{itemize}


\subsection{Inverse DFT}
\begin{align*}
	IDFT(X) = \frac{1}{N}\left[DFT(X^*)\right]^*
	\label{eq:IDFT}
\end{align*}

\subsection{Sampling of Periodic Signals and the DFT}

\subsection{FFT\buchSeite{504-511}}

\subsection{Fast Convolution}
\subsubsection{Circular Convolution\buchSeite{516-518}}
\begin{flalign*}
& \text{Grundlage:}&& y=h\ast x && \Leftrightarrow && Y(\omega)=H(\omega)X(\omega)&&
\end{flalign*}
\begin{flalign*}
& \text{mittels DTFT:} && y=IDTFT(DTFT(h)DTFT(x)&&\\
& \text{mittels DFT:} && \tilde{y}=\widetilde{h\ast x}=IDFT(DFT(h)DFT(x)&&\\
&\text{Modulo-N-reduzierte Ergebnisse}&& \tilde{y}=\widetilde{h\ast x}=\widetilde{\tilde{h}\ast {x}}=\widetilde{h\ast \tilde{x}}=\widetilde{\tilde{h}\ast \tilde{x}}&&\\ % ToDo irgend ein problem mit der widetilde, geht auch mit anderen buchstaben die nicht hoch sind nicht z.b. a
& \text{Bedingung, zirkuläre Faltung = Lineare Faltung} && \tilde{y}=y \text{ wenn } N\geq L_y=L+M&&
\end{flalign*}
\subsubsection{Overlap-Add and Overlap-Save Methods\buchSeite{520-522}}


