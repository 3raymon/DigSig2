\section{DFT/FFT Algorithms\buch{Chapter 9}}
\subsection{Begriffe}

\begin{tabularx}{\textwidth}{l p{3cm}XX}
Abkürzung & Name  & Eigenschaft & Anwendung \\\hline
DTFT & discrete-time Fourier transform & bildet ein endliches, zeitdiskretes Signal auf ein kontinuierliches, periodisches Frequenzspektrum ab & Im Spektrum lassen sich unter umständen  abschnittsweise ein geschlossener mathematischer Ausdruck angeben\\
DFT & discrete Fourier transform & bildet ein zeitdiskretes, endliches Signal, welches periodisch fortgesetzt wird, auf ein diskretes, periodisches Frequenzspektrum ab & DFT besitzt zur Signalanalyse große Bedeutung\\
FFT & fast Fourier transform & ein Algorithmus zur effizienten Berechnung der Werte einer diskreten Fourier-Transformation (DFT) &
\vspace{-19pt}
\begin{itemize}
\item Zur Reduzierung des Berechnungsaufwandes bei der zirkularen Faltung
\item im Zeitbereich von FIR-Filtern
\item Ersatz durch die schnelle Fouriertransformation und einfache Multiplikationen im Frequenzbereich
\item Synthese von Audiosignalen aus einzelnen Frequenzen über die inverse FFT
\end{itemize}\\
\end{tabularx}

\subsection{Frequency Resolution and Windowing\buchSeite{464-472}}
\begin{tabular}{|l|l|l|}
	\hline
	symbol & property & formulas
	\\ \hline
	$T_L$ & duration of data record & $ = LT$
	\\ \hline
	$w(n)$ & Window function &
	\\ \hline
	$\Delta \omega_w$ & Window width (rad/sample) & $ = c \frac{2 \pi}{L}$
	\\ \hline
	$\Delta f_w$ & Window width (Hz) & $ = c \frac{f_s}{L} = c \frac{1}{T_L} $
	\\ \hline
	$\Delta f$ & Frequency resolution & $ \Delta f \geq \Delta f_w$
	\\ \hline
	$R$ & relative sidelobe level &
	\\ \hline
	$c$ & depends on window, always $\geq 1$ &
	\\ \hline
\end{tabular}


\begin{tabularx}{\textwidth}{|l|l|l|l|X|}
	\hline
	\textbf{Window} & $w(n) = $ & $\Delta\omega_w =$ & $\Delta f_w = $ & $\Delta f$
	\\ \hline
	Rectangular &
	$ \begin{cases}
			1, & \text{if } 0 \leq n \leq L-1 \\
			0, & \text{otherwise}
	  \end{cases}$ &
	$ \frac{2\pi}{L}$ &
	$ \frac{f_s}{L} = \frac{1}{LT} = \frac{1}{T_L}$ &
	$ \Delta f \geq \Delta f_w = \frac{f_s}{L}$
	\\ \hline
	Hamming &
	$ \begin{cases}
		0.54 - 0.46 \cos(\frac{2\pi n}{L-1}), & \text{if } 0 \leq n \leq L-1 \\
		0, & \text{otherwise}
	  \end{cases}$ &
	$\frac{4\pi}{L}$ &
	$\frac{2f_s}{L} = \frac{2}{T_L} $ &
	$\Delta f \geq \Delta f_w $
	\\ \hline
\end{tabularx}

\subsection{DTFT Computation}
\subsubsection{DTFT at a Single Frequency\buchSeite{475-477}}
\subsubsection{DTFT over a Frequency Range\buchSeite{478}}
\subsubsection{DFT\buchSeite{479-481}}
\subsubsection{Zero Padding\buchSeite{482}}

\subsection{Physical versus Computional Resolution\buchSeite{482-485}}

\subsection{Matrix Form of DFT}

\subsection{Module-N reduction}
\begin{itemize}
	\item Opposite of zero padding
\end{itemize}


\subsection{Inverse DFT}
\begin{align*}
	IDFT(X) = \frac{1}{N}\left[DFT(X^*)\right]^*
	\label{eq:IDFT}
\end{align*}

\subsection{Sampling of Periodic Signals and the DFT}

\subsection{FFT\buchSeite{504-511}}

\subsection{Fast Convolution}
\subsubsection{Circular Convolution\buchSeite{516-518}}
\begin{flalign*}
& \text{Grundlage:}&& y=h\ast x && \Leftrightarrow && Y(\omega)=H(\omega)X(\omega)&&
\end{flalign*}
\begin{flalign*}
& \text{mittels DTFT:} && y=IDTFT(DTFT(h)DTFT(x)&&\\
& \text{mittels DFT:} && \tilde{y}=\widetilde{h\ast x}=IDFT(DFT(h)DFT(x)&&\\
&\text{Modulo-N-reduzierte Ergebnisse}&& \tilde{y}=\widetilde{h\ast x}=\widetilde{\tilde{h}\ast {x}}=\widetilde{h\ast \tilde{x}}=\widetilde{\tilde{h}\ast \tilde{x}}&&\\ % ToDo irgend ein problem mit der widetilde, geht auch mit anderen buchstaben die nicht hoch sind nicht z.b. a
& \text{Bedingung, zirkuläre Faltung = Lineare Faltung} && \tilde{y}=y \text{ wenn } N\geq L_y=L+M&&
\end{flalign*}
\subsubsection{Overlap-Add and Overlap-Save Methods\buchSeite{520-522}}


