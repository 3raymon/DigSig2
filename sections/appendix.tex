
\appendix

\section{DFT Matrices}\label{app:DFT}

\begin{align*}
	W_2 = e^{-2\pi j / 2} = -1 \\
	A_2 &= 
		\begin{bmatrix}
			1& 1 \\
			1& W_2 \\
		\end{bmatrix}
		&&= 
		\begin{bmatrix}
			1& 1 \\
			1& -1 \\
		\end{bmatrix}	 \\
	W_4 = e^{-2\pi j / 4} = \cos(\pi/2) - j\sin(\pi/2) = -j \\
	A_4 &= 
		\begin{bmatrix}
			1& 1& 1& 1 \\
			1& W_4& W_4^2& W_4^3 \\
			1& W_4^2& W_4^4& W_4^6 \\
			1& W_4^3& W_4^6& W_4^9 \\
		\end{bmatrix}
		&&=
		\begin{bmatrix}
			1& 1& 1& 1 \\
			1& -j& -1& j \\
			1& -1& 1& -1 \\
			1& j& -1& -j \\
		\end{bmatrix} \\
	W_8 = e^{-2\pi j / 8} = \cos(\pi/4) - j\sin(\pi/4) = \frac{1-j}{\sqrt{2}} \\
	A_8 &= 
		\begin{bmatrix}
			1 & 1 & 1 & \dots & 1 \\
			1 & W_8 & W_8^2 & \dots & W_8^7 \\
			1 & W_8^2 & W_8^4 & \dots & W_8^{14} \\
			1 & W_8^3 & W_8^6 & \dots & W_8^{21} \\
			\vdots & \vdots & \vdots & \ddots & \vdots \\
			1 & W_8^7 & W_8^{14} & \vdots & W_8^{49} \\
		\end{bmatrix}
		%TODO wer lustich ist darf das gerne noch vervollständigen
\end{align*}

\section{Butterworth polynomials}\label{app:Butterworth}

\begin{tabular}{|l|l|l|l|}
	\hline
	N & K & $\theta_1,\theta_2,\ldots,\theta_K$ & $D(s)$ \\ \hline
	1 & 0 & & $1+s$ \\ \hline
	2 & 1 & $\frac{3\pi}{4}$ & $(1 + 1.4142s + s^2)$ \\ \hline
	3 & 1 & $\frac{4\pi}{6}$ & $(1+s)(1+s+s^2)$ \\ \hline
	4 & 2 & $\frac{5\pi}{8}\:,\:\frac{7\pi}{8}$ & $(1+0.7654s+s^2)(1+1.8478s+s^2)$ \\ \hline
	5 & 2 & $\frac{6\pi}{10}\:,\:\frac{8\pi}{10}$ & $(1+s)(1+0.6180s+s^2)(1+1.6180s+s^2)$ \\ \hline
	6 & 3 & $\frac{7\pi}{12}\:,\:\frac{9\pi}{12}\:,\:\frac{11\pi}{12}$ & $(1+0.5176s+s^2)(1+1.4142s+s^2)(1+1.9319s+s^2)$ \\ \hline
	7 & 3 & $\frac{8\pi}{14}\:,\:\frac{10\pi}{14}\:,\:\frac{12\pi}{14}$ & $(1+s)(1+0.4450s+s^2)(1+1.2470s+s^2)(1+1.8019s+s^2)$ \\ \hline
\end{tabular}

\newpage
\section{Idiotenseite}\label{app:Idiotenseite}
\addtocontents{toc}{\protect\setcounter{tocdepth}{-1}}
\input{idiotenseite/trigo/subsections/Winkelargumente}
\input{idiotenseite/trigo/subsections/Quadrantenbeziehungen}
\input{idiotenseite/trigo/subsections/Periodizitaet}
\begin{multicols}{2}
\input{idiotenseite/trigo/subsections/Additionstheoreme}
\input{idiotenseite/trigo/subsections/DoppelHalbwinkel}
\input{idiotenseite/trigo/subsections/Produkte}
\input{idiotenseite/trigo/subsections/SummeDifferenzen}
\input{idiotenseite/trigo/subsections/Euler}

\end{multicols}
\input{idiotenseite/diverses/subsections/Reihenentwicklung}

