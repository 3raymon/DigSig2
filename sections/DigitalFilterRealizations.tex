\section{Digital Filter Realizations\buch{Chapter 7}}
\subsection{Direct Form \buchSeite{265}}
\label{sec:directform}
Auch \emph{direct form I} realization genannt.

Consider a second-order filter with transfer function
\begin{flalign*}
	H(z) = \frac{N(z)}{D(z)} 
		 = \frac{b_0 + b_1z^{-1} + b_2z^{-2}}{1 + a_1z^{-1} + a_2z^{-2}}&&
\end{flalign*}
\begin{flalign*}
&\text{I/O differenzen Gleichung} && y_n=-a_1y_{n-1}-a_2y_{n-2}-\ldots-a_My_{n-M}+b_0x_n+b_1x_{n-1}+\ldots+b_Mx_{n-M}
\end{flalign*}\\

% FIR filter as block diagram
\begin{tikzpicture}
	% Place nodes using a matrix
	\matrix (m1) [row sep=2.5mm, column sep=5mm]
	{
		%--------------------------------------------------------------------
		\node[dspnodeopen,dsp/label=above] (m00) {$x(n)$};    &
		\node[coordinate]                  (m01) {};          &
		\node[dspmixer, dsp/label=below]   (m02) {$b_0$};     &
		
		\node[coordinate]                  (m025) {};          &
		\node[dspadder]                    (m03) {};          &
		\node[coordinate]                  (m035) {};          &
		
		\node[coordinate]                  (m04) {};          &
		\node[coordinate]                  (m05) {};          &
		\node[dspnodeopen,dsp/label=above] (m06) {$y(n)$};    \\
		
		\node[coordinate]                  (m10) {};          &
		\node[dspsquare]                   (m11) {$\z^{-1}$}; &
		\node[coordinate]                  (m12) {};          &
		
		\node[coordinate]                  (m125) {};          &
		\node[coordinate]                  (m13) {};          &
		\node[coordinate]                  (m135) {};          &
		
		\node[coordinate]                  (m14) {};          &
		\node[dspsquare]                   (m15) {$\z^{-1}$}; &
		\node[coordinate]                  (m16) {};          \\
		
		\node[coordinate]                  (m20) {};          &
		\node[coordinate]                  (m21) {};          &
		\node[dspmixer, dsp/label=below]   (m22) {$b_1$};     &
		
		\node[coordinate]                  (m225) {};          &
		\node[coordinate]                  (m23) {};          &
		\node[coordinate]                  (m235) {};          &
		
		\node[dspmixer, dsp/label=below]   (m24) {$-a_1$};     &
		\node[coordinate]                  (m25) {};          &
		\node[coordinate]                  (m26) {};          \\
		
		\node[coordinate]                  (m30) {};          &
		\node[dspsquare]                   (m31) {$\z^{-1}$}; &
		\node[coordinate]                  (m32) {};          &
		
		\node[coordinate]                  (m325) {};          &
		\node[coordinate]                  (m33) {};          &
		\node[coordinate]                  (m335) {};          &
		
		\node[coordinate]                  (m34) {};          &
		\node[dspsquare]                   (m35) {$\z^{-1}$}; &
		\node[coordinate]                  (m36) {};          \\
		
		\node[coordinate]                  (m40) {};          &
		\node[coordinate]                  (m41) {};          &
		\node[dspmixer, dsp/label=below]   (m42) {$b_2$};     &
		
		\node[coordinate]                  (m425) {};          &
		\node[coordinate]                  (m43) {};          &
		\node[coordinate]                  (m435) {};          &
		
		\node[dspmixer, dsp/label=below]   (m44) {$-a_2$};     &
		\node[coordinate]                  (m45) {};          &
		\node[coordinate]                  (m46) {};          \\
		};
		
	% Draw connections
	\begin{scope}[start chain]
		\chainin (m00);
		\chainin (m02) [join=by dspconn];
		\chainin (m03) [join=by dspconn];
		\chainin (m06) [join=by dspconn];
	\end{scope}
	
	\begin{scope}[start chain]
		\chainin (m01);
		\chainin (m11) [join=by dspconn];
		\chainin (m31) [join=by dspconn];
		\chainin (m41) [join=by dspline];
		\chainin (m42) [join=by dspconn];
		\chainin (m425) [join=by dspline];
		\chainin (m03) [join=by dspconn];
	\end{scope}
	
	\begin{scope}[start chain]
		\chainin (m21);
		\chainin (m22) [join=by dspconn];
		\chainin (m03) [join=by dspconn];
	\end{scope}
	
	\begin{scope}[start chain]
		\chainin (m05);
		\chainin (m15) [join=by dspconn];
		\chainin (m35) [join=by dspconn];
		\chainin (m45) [join=by dspline];
		\chainin (m44) [join=by dspconn];
		\chainin (m435) [join=by dspline];
		\chainin (m03) [join=by dspconn];
	\end{scope}
	
	\begin{scope}[start chain]
		\chainin (m25);
		\chainin (m24) [join=by dspconn];
		\chainin (m03) [join=by dspconn];
	\end{scope}
	
	\draw[] (m01) -- node[midway,left] {$v_0$} (m11);
	\draw[] (m11) -- node[midway,left] {$v_1$} (m31);
	\draw[] (m31) -- node[midway,left] {$v_2$} (m41);
			
	\draw[] (m05) -- node[midway,right] {$w_0$} (m15);
	\draw[] (m15) -- node[midway,right] {$w_1$} (m35);
	\draw[] (m35) -- node[midway,right] {$w_2$} (m45);
\end{tikzpicture}


\subsection{Canonical Form\buchSeite{271}}
\label{sec:canonicalform}
Auch \emph{direct form II} genannt.

It can be obtained from the \nameref{sec:directform} by splitting it up and reversing it.
\begin{flalign*}
H(z) = \frac{1}{D(z)} \cdot N(z)&&
\end{flalign*}

\begin{flalign*}
&\text{I/O differenzen Gleichung} && w(n)=x(n)-a_1w(n-1)-\ldots-a_Mw(n-M)&&\\&&& y(n)=b_0w(n)+b_1w(n-1)+\ldots+b_Lw(n-L)&&
\end{flalign*}\\
% FIR filter as block diagram
\begin{tikzpicture}
	% Place nodes using a matrix
	\matrix (m) [row sep=2.5mm, column sep=5mm]
	{
		%--------------------------------------------------------------------
		\node[dspnodeopen,dsp/label=above] (m00) {$x(n)$};    &
		\node[dspadder]                    (m01) {};          &
		\node[coordinate]                  (m02) {};     &
		\node[coordinate]                  (m03) {};          &
		\node[dspmixer, dsp/label=below]   (m04) {$b_0$};     &
		\node[dspadder]                    (m05) {};          &
		\node[dspnodeopen,dsp/label=above] (m06) {$y(n)$};    \\
	
		
		\node[coordinate]                  (m10) {};          &
		\node[coordinate]                  (m11) {};          &
		\node[coordinate]                  (m12) {};          &
		\node[dspsquare]                   (m13) {$\z^{-1}$}; &
		\node[coordinate]                  (m14) {};          &
		\node[coordinate]                  (m15) {};          &
		\node[coordinate]                  (m16) {};          \\
		
		\node[coordinate]                  (m20) {};          &
		\node[coordinate]                  (m21) {};          &
		\node[dspmixer, dsp/label=below]   (m22) {$-a_1$};     &
		\node[coordinate]                  (m23) {};          &
		\node[dspmixer, dsp/label=below]   (m24) {$b_1$};     &
		\node[coordinate]                  (m25) {};          &
		\node[coordinate]                  (m26) {};          \\
		
		
		\node[coordinate]                  (m30) {};          &
		\node[coordinate]                  (m31) {};          &
		\node[coordinate]                  (m32) {};          &
		\node[dspsquare]                   (m33) {$\z^{-1}$}; &
		\node[coordinate]                  (m34) {};          &
		\node[coordinate]                  (m35) {};          &
		\node[coordinate]                  (m36) {};          \\
		
		\node[coordinate]                  (m40) {};          &
		\node[coordinate]                  (m41) {};          &
		\node[dspmixer, dsp/label=below]   (m42) {$-a_2$};     &
		\node[coordinate]                  (m43) {};          &
		\node[dspmixer, dsp/label=below]   (m44) {$b_2$};     &
		\node[coordinate]                  (m45) {};          &
		\node[coordinate]                  (m46) {};          \\
		};
		
	% Draw connections
	\begin{scope}[start chain]
		\chainin (m00);
		\chainin (m01) [join=by dspconn];
		\chainin (m04) [join=by dspconn];
		\chainin (m05) [join=by dspconn];
		\chainin (m06) [join=by dspconn];
	\end{scope}
	
	\begin{scope}[start chain]
		\chainin (m03);
		\chainin (m13) [join=by dspconn];
		\chainin (m23) [join=by dspline];
		\chainin (m33) [join=by dspconn];
		\chainin (m43) [join=by dspline];
	\end{scope}
	
	\foreach \i in {2,4}
	{
		\begin{scope}[start chain]
			\chainin (m\i3);
			\chainin (m\i2) [join=by dspconn];
			\chainin (m01) [join=by dspconn];
		\end{scope}
	}
	
	\foreach \i in {2,4}
	{
		\begin{scope}[start chain]
			\chainin (m\i3);
			\chainin (m\i4) [join=by dspconn];
			\chainin (m05) [join=by dspconn];
		\end{scope}
	}
		
	\draw[] (m01) -- node[midway,above] {$w_0$} (m04);
	\draw[] (m23) -- node[midway,above] {$w_1$} (m24);
	\draw[] (m43) -- node[midway,above] {$w_2$} (m44);
\end{tikzpicture}


\subsection{Cascade Form\buchSeite{277}}
\label{sec:cascadeform}
Die Kaskadenform verwendet nur Transferfunktionen 2. Ordnung.

\begin{flalign*}
H(z) = \prod_{i=0}^{K-1}H_i(z) = \prod_{i=0}^{K-1}\frac
{b_{i0}+b_{i1}z^{-1}+b_{i2}z^{-2}}
{1+a_{i1}z^{-1}+a_{i2}z^{-2}} &&
\end{flalign*}

Dies ist von Vorteil, da sich Transferfunktionen 2. Ordung effizient implementieren lassen.

\subsection{Cascade to Canonical\buchSeite{284}}
Um die \nameref{sec:directform} oder \nameref{sec:canonicalform} in die \nameref{sec:cascadeform} umzuwandeln, müssen $D(z)$ und $N(z)$ in Faktoren 2. Ordnung faktorisiert werden.
\begin{flalign*}
D(z) &= 1 + a_1 z^{-1} + a_2 z^{-2} + \dots + a_M z^{-M} &\\
&=(1-p_1z^{-1})(1-p_2z^{-1}) \dots (1-p_Mz^{-1})&
\end{flalign*}

\subsection{Hardware Realizations and Circular Buffers\buchSeite{293}}
% Brauchts diese Section?
